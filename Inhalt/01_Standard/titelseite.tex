\thispagestyle{empty}
\begin{titlepage}
\enlargethispage{4cm}

\begin{figure}           % Logo vom Ausbildungsbetrieb und der DHBW
	\vspace*{-5mm}       % Sollte dein Titel zu lang werden, kannst du mit diesem "Hack" 
	%                      den Inhalt der Seite nach oben schieben.
	\begin{minipage}{0.49\textwidth}
		\flushleft
		\includegraphics[height=2.5cm]{Bilder/Logos/Logo_SAP.pdf} 
	\end{minipage}
	\hfill
	\begin{minipage}{0.49\textwidth}
		\flushright
		\includegraphics[height=2.5cm]{Bilder/Logos/Logo_DHBW.pdf} 
	\end{minipage}
\end{figure} 
\vspace*{0.1cm}

\begin{center}
	\begin{spacing}{0.9}
		\huge{\textbf{\titel}}\\[1.5cm]
	\end{spacing}
	\Large{\textbf{\arbeit}}\\[0.5cm]
	\normalsize{im Rahmen der Prüfung zum\\[1ex] \textbf{Bachelor of Science (B.Sc.)}}\\[0.5cm]
	\Large{des Studienganges \studiengang}\\[1ex]
	\normalsize{an der Dualen Hochschule Baden-Württemberg Karlsruhe}\\[1cm]
	\normalsize{von}\\[1ex] \Large{\textbf{\autor}} \\[1cm]

	% Hinweis: Manche Dozenten möchten einen Hinweis auf den Sperrvermerk auf der Titelseite.

	% Sperrvermerkt ein-/auskommentieren:
	\large{{\color{red}- Sperrvermerk -}}\\[1cm]


\end{center}

\begin{center}
	\vfill
	\begin{tabular}{ll}
		Abgabedatum:                     & \abgabe \\[0.2cm]
		Bearbeitungszeitraum:            & \bearbeitungszeitraum \\[0.2cm]
		Kurs:            				 & \kurs \\[0.2cm]
		Ausbildungsfirma:                & \firmaName \\
		                                 & \firmaStrasse \\
		                                 & \firmaPlz \\[0.2cm]
		Betreuer der Ausbildungsfirma:   & \betreuerFirma \\[0.2cm]
		Gutachter der Dualen Hochschule: & \betreuerDhbw \\[2cm]
	\end{tabular} 
\end{center}
\end{titlepage}
