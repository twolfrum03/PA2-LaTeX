\chapter{Schlussbetrachtungen}

\section{Zusammenfassung}

-> relevante Punkte der Arbeit zusammenfassen, damit der Leser im nächsten Kapitel der Handlungsempfehlung die wichtigsten Punkte vor Augen hat

\section{Handlungsempfehlung}

-> Hier soll auf Basis der Evaluation der Lösungsansätze anhand der Kriterien eine Handlungsempfehlung gegeben werden, wie der Prozess umgesetzt werden sollte, damit für den Kunden ein Mehrwert entsteht

\section{kritische Reflexion und Ausblick}

-> Zuletzt soll sich hier nochmal kritisch mit der Arbeit auseinandergesetzt werden, eventuell nicht berücksichtigte Punkte, etc. angesprochen werden, damit der Leser ein realistisches Bild erhält

-> Der Ausblick soll das weitere Vorgehen im Kundenprojekt beschreiben (wie die Erkentnisse der Arbeit angewandt werden) und ggf. weiteres Forschungspotenzial auf einem allgemeineren Level aufzeigen