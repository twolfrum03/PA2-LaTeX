\chapter{Einleitung}

%Umfang: ca. 2-3 Seiten

\section{Motivation und Problemstellung}

Durch die Digitalisierung, zunehmende Komplexität globaler Lieferketten, starken Preisdruck der Konkurrenz und dem Wechsel zur nachhaltigen Mobilität befindet sich die Automobilbranche in einem bedeutenden Wandel. Der Einkauf ist seither ein gro\ss er Hebel, um die Produktionskosten zu senken und dadurch die Margen erhöhen zu können. Deshalb kommt einem optimalen Beschaffungssystem eine immer grö\ss ere Bedeutung zu. Ein wichtiger Bestandteil ist hierbei ein effizientes Massendaten-Management, da Einkäufer vor der Herausforderung stehen, die Datenmengen, die mit der komplexen Beschaffung vieler Teile einhergehen, zu bewältigen. 

Im Kontext des ''connected Procurement''-Beratungsprojekts möchte der deutsche Automobilhersteller BMW die SAP Produktsuite für die direkte Materialbeschaffung auf Basis von S/4 HANA einführen und somit Einkaufsprozesse digitalisieren und optimieren. Unter anderem soll die Zusammenarbeit mit Lieferanten in zentralen Einkaufskontrakten verwaltet werden. Die Preisbestandteile einzelner zu beschaffender  Bauteile müssen in diesen Central Contracts jährlich im Rahmen der Preisverhandlungen aktualisiert werden. Dafür wird eine Möglichkeit, um die Verträge effizient in Masse zu bearbeiten benötigt, da die Benutzung der SAP-Standardfunktionalität aufgrund von hohem Zeitaufwand und Fehleranfälligkeit nicht infrage kommt. Aufgrund der strategischen Relevanz des Kunden und dessen hoher Priorisierung des Prozesses wird zudem eine Übernahme der entwickelten Lösung in den SAP-Standard in Betracht gezogen.

\section{Ziel der Arbeit}

Das Ziel dieser Arbeit ist es, eine Handlungsempfehlung für den Kunden zu auszusprechen, wie der Massenbearbeitungsprozess von Zentralkontrakten innerhalb des SAP-Produkts optimiert werden kann. Es soll die Frage beantwortet werden, wie der Prozess am Besten gestaltet und umgesetzt werden kann, um für den Endanwender eine effiziente Bearbeitung der Verträge mit einer hohen Benutzerfreundlichkeit zu ermöglichen. Dies soll durch die Analyse des Ist-Zustands und der anschlie\ss enden Konzeption zweier Lösungsmöglichkeiten anhand der Anforderungen des Kunden ermöglicht werden. Durch die Bewertung beider Lösungen sollen die Vor- und Nachteile letzterer herausgearbeitet und dadurch eine Empfehlung für die Abbildung des Prozesses gegeben werden.  

% -> Wichtigster Teil der Einleitung (Ziel der Arbeit in 1. Satz auf den Punkt bringen, danach mehr ausführen, hier Forschungsfrage rein)
% -> Das Ziel der Arbeit muss bei direktem Vergleich stimmig mit dem Fazit sein!

\section{Thematische Abgrenzung}

Die vorliegende Arbeit fokussiert sich auf der Prozessebene die direkte Beschaffung von komplexen Bauteilen in der Automobilindustrie und die damit verbundenen Prozesse. Da die Arbeit im Kontext eines Beratungsengagements der SAP SE bei BMW entstanden ist und aufgrund der IT-Strategie des Kunden eine homogene Systemlandschaft angestrebt wird, werden lediglich Umsetzungsmöglichkeiten innerhalb des SAP-Produktportfolios betrachtet. Andere Anbieter und Lösungen werden nicht berücksichtigt, da diese beispielsweise nach Gesichtspunkten der Integration und Masterdatenverfügbarkeit nicht zielführend wären. 

Zudem wird funktional im Allgemeinen die massenhafte Bearbeitung von zentralen Einkaufsverträgen und im Speziellen die Bearbeitung verschiedener Preisbestandteile betrachtet. Weitere Aspekte des Central Contracts werden über Standardfunktionalitäten abgedeckt und sind somit nicht Gegenstand dieser Arbeit.

Aufgrund des limitierten Umfangs der Arbeit wird sich auf die Konzeption der Lösungen beschränkt. Letztere werden lediglich prototypisch umgesetzt, jedoch nicht vollständig implementiert.

% -> Weiterer Abstraktionsgrad auf generelles SAP-Umfeld oder generelles Massendatenmanagement wäre schön für wissenschaftliche Relevanz, aber nur soweit es Thema zulässt, wenn nicht möglich muss das gut begründet werden

\section{Methodisches Vorgehen}

In dieser Arbeit wurde die Methode des Experteninterviews angewende, um die Anforderungen des Kunden an den Prozess der Massenbearbeitung von Zentralen Einkaufsverträgen zu ermitteln. Zu diesem Zweck wurde der Projektmanager für das globale Beschaffungssystem und Verantwortlicher für die Einkaufsprozesse bei BMW interviewt. Experteninterviews ermöglichen in diesem Zusammenhang einen umfassenden Erkenntnisgewinn und Zugang zu praxisrelevanten Informationen, die so in der Literatur nicht verfügbar sind, da es sich um kundenspezifische Anforderungen handelt. Bei der Durchführung wurde eine unstrukturierte Vorgehensweise gewählt, da das allgemeine Ziel die Anforderungsermittlung war und die Anforderungen aus einer unvoreingenommener und nicht durch vordefinierte Fragen eingeschränkter Kundenperspektive aufgenommen werden sollten. Dennoch wurden bei Bedarf in einzelnen Bereichen Rückfragen gestellt, um gezielte Informationen zu erhalten.

Die Bewertung der in der Arbeit vorgestellten Lösungen erfolgt durch eine Nutzwertanalyse. Diese Methode ermöglicht eine quantitative Bewertung der Lösungen anhand von Kriterien, die im Vorfeld definiert wurden. Die Kriterien wurden in Abstimmung mit dem Kunden festgelegt und sollen die Anforderungen des Kunden an die Lösung widerspiegeln. Die Bewertung der Kriterien erfolgt durch den Autor der Arbeit auf einer Skala von eins bis fünf Punkten, woraus eine Gesamtbewertung der Lösungen ermittelt. Somit kann eine fundierte Handlungsempfehlung gegeben werden.