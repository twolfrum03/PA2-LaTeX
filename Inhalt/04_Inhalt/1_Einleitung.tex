\chapter{Einleitung}

\section{Unternehmensprofil und Anwendungsbezug}

SAP SE ist ein börsennotierter Softwarekonzern mit Sitz in Walldorf. Das Hauptgeschäft des 1972 gegründeten Unternehmens ist die Entwicklung von Unternehmenssoftware zur Abwicklung von Geschäftsprozessen. Heute erwirtschaften 105.000 Mitarbeiter in 157 Ländern einen Umsatz von ca. 30 Mrd. \euro{}. Erfolgreich wurde das Unternehmen mit dem Verkauf von ERP Standardsoftware. In den letzten Jahren stand die Transformation des gesamten Produkt-Portfolios in Richtung Cloud-Services als Abo-Modell im Fokus der Unternehmensstrategie. \footcite[Vgl.][]{sap_geschichte_2023}

Die Abteilung AIS HCM ist Teil des Unternehmensbereichs Product Engineering und zuständig für den Development-Support und Neuentwicklungen der SAP Personallösung HCM. Diese deckt Prozesse rund um das Personalwesen ab. Zudem stellt die Abteilung mehrere SAP Fiori Apps als Self-Service für Mitarbeiter bereit. Durch das hohe Nutzungsvolumen dieser Apps und der somit gro{\ss}en betriebswirtschaftlichen Relevanz, sind diese und auch der Untersuchungsgegenstand dieser Arbeit für das Produkt HCM von gro{\ss}er Bedeutung.

% - SAP-Personallösung HCM als Teil des ERP-Systems ECC (alt) und S/4 HANA (neu)
% - Abdeckung Software rund um Personalwesen , -planung (Abwesenheiten, Zeiten)
% - Keine Unternehmenshierarchie
% - Abteilung: Development-Suppport, Analyse Systemlandschaft, Dimensionierung
% - ECC, S4 on-prem Editionen -> Wartung für Kunden mit Wartungsverträgen, Neuentwicklung für HCM
% - Arbeit im Kontext der SAP HCM Anwendung, die Teil von SAP ECC, S/4 ist

\section{Motivation und Problemstellung}

Die von der Abteilung betriebenen Fiori Apps, die schon im Zusammenhang der Einleitung angesprochen wurden, sind auf Basis des Frameworks SAP UI5 Freestyle für ein älteres Produkt - SAP ECC - entwickelt worden. Durch die strategische Entscheidung HCM als Bestandteil von S/4 zu integrieren, finden die S/4-Design-Guidelines darauf Anwendung, die \zB Oberflächen-Design oder zu verwendende Technologien, festlegen. Fiori Apps müssen dadurch die Technologie Fiori Elements verwenden. Existierende Apps werden auf Basis der älteren Technologie in S/4 weiterbetrieben werden und neu entwickelte Apps müssen Fiori Elements verwenden.

Diese Situation sorgt für ein Problem in Geschäftsprozessen, die über solche Apps abgebildet werden sollen. Das Framework Fiori Elements generiert das gesamte Front-End der Anwendung selbstständig. Dadurch ist Das erleichtert auf der einen Seite die Entwicklung der Apps, auf der anderen Seite kann dadurch keine eigene Programmlogik mehr im Front-End eingebaut werden. Zudem bietet das Programmiermodell RAP, das in Fiori Elements für eine konsistente Durchführung der Datenbankoperationen zuständig ist, nur einen transaktionalen Kontext für das Ausführen von eigener Logik. Jedoch ist es in bestimmten Geschäftsprozessen nötig, nachgelagert in einem weiteren transaktionalen Kontext noch Programmcode ausführen zu können. Da RAP das modellseitig nicht zulässt, soll in der vorliegenden Arbeit soll nun untersucht werden, wie sich solche asynchronen Prozesse, trotz den eben dargelegten Einschränkungen im S/4 Umfeld mit den neueren Technologien umsetzen lassen. asdf

% - SAP UI5 ist open-source UI-Bibliothek -> Fiori als darauf aufbauendes Framework zur Entwicklung von Apps
% - Fiori gibt es als Freestyle und in S/4 als Elements Variante
% - Elements sehr starr, nur ein transaktionaler Kontext, daneben keine weitere asynchrone Code-Ausführung ->Problem
% - RAP stellt im Kontext der Fiori App sicher, dass Datenbankoperationen innerhalb einem LuW konsistent durchgeführt wird
% - Brauchen nebem "Haupt LUW" noch anderen LUW der zb Emails versendet oder asynchron noch Code ausführt -> geht nicht in RAP/ Fiori Elements ->Aufruf von "2. Programm"
% - Transaktionaler Kontext: LuW
% - keine Treiber der strategischen Entscheidung nennen
% - "HCM ist in S/4, deswegen Frameworks, ..., deswegen Probleme

\section{Aufbau und Ziel der Arbeit}

Am Anfang wird SAP und die Abteilung AIS HCM, in der die Praxisphase durchgeführt wurde, vorgestellt, um den Kontext der Arbeit zu erläutern. Die Motivation und Problemstellung werden ebenfalls diskutiert und die Arbeit klar abgegrenzt, sowie die wissenschaftlichen Methoden kurz genannt. Der theoretische Teil befasst sich mit der Darstellung der theoretischen Grundlagen einer RESTful API und deren Umsetzung im RESTful Application Programming Model der SAP, sowie die Erläuterung der Technologie Fiori Elements. Es folgt die Vorstellung und Gegenüberstellung von drei Technologien – Business Workflows, Business Events und dem Background Processing Framework – zur Lösung der in der Problemstellung adressierten Frage. Anhand von Kriterien wie Stärken und Schwächen, Effizienz und Robustheit werden die Technologien verglichen und in einer Entscheidungsmatrix dargestellt. Die Matrix soll als Orientierungshilfe dienen, welche Technologie sich am besten zur Abbildung asynchroner Prozesse in einem RESTful API-Umfeld eignet. Abschließend werden die Ergebnisse zusammengefasst und kritisch reflektiert, Handlungsempfehlungen gegeben und ein Ausblick auf zukünftige Entwicklungen präsentiert.

\section{Abgrenzung}

Der Zweck der vorliegenden Arbeit ist es, die drei vorgestellten Technologien vergleichend zu bewerten und je nach Anwendungsfall eine Handlungsempfehlung im Bezug auf eine sich anbietende Technologie zu geben. Über diese drei Technologien hinaus werden keine anderen Möglichkeiten asynchrone Prozesse abzubilden, wie \zB im Cloud Application Programming Model (CAP) behandelt. Au{\ss}erdem findet aufgrund des beschränkten Umfangs der Arbeit lediglich ein Vergleich der Technologien statt und keine direkte Implementierung dieser in einem konkreten Anwendungsfall. Hierfür sei auf die offizielle Dokumentation der SAP mit Showcases für die respektiven Technologien verwiesen.

\section{Methodisches Vorgehen}

Die Vergleichskriterien werden bei den einzelnen Ansätzen durch Experteninterviews bewertet und dann die betrachteten Ansätze anhand dieser Kriterien gegenübergestellt. So kommt dann die Entscheidungsmatrix zustande, welche Technologie sich bei welchen Anforderungen und Rahmenbedingungen anbietet.

In dieser Arbeit wurde die Methode der Experteninterviews angewendet, um spezifische SAP-Technologien zu untersuchen und in Bezug auf vordefinierte Vergleichskriterien zu bewerten. Experteninterviews ermöglichen eine umfassende Erkenntnisgewinnung und den Zugang zu praxisrelevanten Informationen, die oft nicht ausreichend in der Fachliteratur abgebildet sind. Bei der Durchführung wurde eine Mischform aus strukturierten und unstrukturierten Interviews gewählt, welche klare Fragestellungen mit Flexibilität bei der Beantwortung verbindet. Dieser Ansatz erlaubte eine Anpassung an individuelle Anwendungsfälle und trug zur Identifizierung von Best Practices und relevanten Herausforderungen in diesem spezifischen SAP-Umfeld bei.