\chapter{Einleitung}

-> Allgemeine Frage: Wie ausführlich soll die Einleitung sein?

\section{Motivation und Problemstellung}

-> Beratungsprojekt bei wichtigem Kunden

-> häufig genutzter Prozess, der sehr umständlich und fehleranfällig ist

-> Von Kundenseite hohe Priorität, dass Prozess in naher Zukunft abgelöst wird

-> Sachnummern in Verträgen mit vielen verschiedenen Konditionen anzulegen ist repetetive und fehleranfällige Arbeit, deshlab muss Weg gefunden werden, wie dieser Prozess optimiert werden kann

->Gründe für Massenänderung: Unternehmen ändert Zahlungsbedingungen, dadurch Änderung aller Kontrakte notwendig; Preisnachverhandlung, dadurch Preisreduktion in allen Kontrakten notwendig

\section{(Aufbau und) Ziel der Arbeit}

-> Aufbau der Arbeit: ggf. weglassen, da im Endeffekt nur Gliederung einmal wiederholt und in Textform ausformuliert wird (es sei denn von dir gewünscht)

-> Ziel der Arbeit: Nach Ist-, Anforderungsanalyse und Konzeption mehrerer Möglichkeiten sollen diese anhand von Kriterien gegeneinander abgewogen werden, um am Ende eine Handlungsempfehlung für den Kunden zu geben

\section{Thematische Abgrenzung}

-> Der Hauptfokus der Arbeit liegt auf der Konzeption (je nachdem wie viel Platz ich noch habe könnte man eine prototypische Umsetzung auch noch mit reinbringen, würde ich aber dynamisch entscheiden)

-> Es geht um den speziellen Use-Case des Massenimports/ -bearbeitung von Contract Line Items in Zentralkontrakten in SAP S/4 HANA Central Procurement (Frage an dich: Soll in der Arbeit auch abstrahiert werden und eine allgemeine Handlungsempfehlung für Massendatenoperationen in SAP gegeben werden?)

\section{Methodisches Vorgehen}

-> Anbieten würden sich Experteninterviews, z.B. mit Experten für den technischen und geschäftlichen Part des Prozesses

-> Zudem könnte man einen Workshop mit Endanwendern durchführen, um Anforderungen an den Prozess zu ermitteln (bin mir noch nicht ganz sicher, in wie weit ich das umsetzen kann)

-> Um ein gutes Verständnis für den Prozess zu schaffen würde es sich noch anbieten den Prozess anfangs zu modellieren (z.B. mit BPMN)