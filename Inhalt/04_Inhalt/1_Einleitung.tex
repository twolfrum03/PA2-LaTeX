\chapter{Einleitung}

Umfang: ca. 2-3 Seiten

\section{Motivation und Problemstellung}

-> Wichtig: Problem auf einen Punkt bringen (wichtiger Prozess, schlechte UX, ...), warum ist die Arbeit wichtig

-> Beratungsprojekt bei wichtigem Kunden

-> häufig genutzter Prozess, der sehr umständlich und fehleranfällig ist

-> Von Kundenseite hohe Priorität, dass Prozess in naher Zukunft abgelöst wird

-> Sachnummern in Verträgen mit vielen verschiedenen Konditionen anzulegen ist repetetive und fehleranfällige Arbeit, deshlab muss Weg gefunden werden, wie dieser Prozess optimiert werden kann

-> Gründe für Massenänderung: Unternehmen ändert Zahlungsbedingungen, dadurch Änderung aller Kontrakte notwendig; Preisnachverhandlung, dadurch Preisreduktion in allen Kontrakten notwendig

-> eventuelle Übernahme in SAP Standad gutes Arguement für Relevanz der Arbeit

\section{Ziel der Arbeit}

-> Wichtigster Teil der Einleitung (Ziel der Arbeit in 1. Satz auf den Punkt bringen, danach mehr ausführen, hier Forschungsfrage rein)

-> Das Ziel der Arbeit muss bei direktem Vergleich stimmig mit dem Fazit sein!

% -> Ziel der Arbeit: Nach Ist-, Anforderungsanalyse und Konzeption mehrerer Möglichkeiten sollen diese anhand von Kriterien gegeneinander abgewogen werden, um am Ende eine Handlungsempfehlung für den Kunden zu geben

\section{Thematische Abgrenzung}

-> Der Hauptfokus der Arbeit liegt auf der Konzeption (je nachdem wie viel Platz ich noch habe könnte man eine prototypische Umsetzung auch noch mit reinbringen, würde ich aber dynamisch entscheiden)

-> Es geht um den speziellen Use-Case des Massenimports/ -bearbeitung von Contract Line Items in Zentralkontrakten in SAP S/4 HANA Central Procurement 

-> Weiterer Abstraktionsgrad auf generelles SAP-Umfeld oder generelles Massendatenmanagement wäre schön für wissenschaftliche Relevanz, aber nur soweit es Thema zulässt, wenn nicht möglich muss das gut begründet werden
\section{Methodisches Vorgehen}

-> Experteninterviews (technischen und geschäftlichen Part des Prozesses)

% -> Zudem könnte man einen Workshop mit Endanwendern durchführen, um Anforderungen an den Prozess zu ermitteln

-> Nutzwertanalyse (quantitative Skala, 5-10 Kriterien, Bewertung selbst möglich, aber müssen begründet werden, Kriterien sollen sich möglichst wenig überlappen, Bewertung kritisch hinterfragen, Dreieich Budget, Zeit, Scope, MECE für Kriterien)

-> Modellieren von Ist- bzw. Soll-Stand des Prozesses (z.B. mit BPMN)