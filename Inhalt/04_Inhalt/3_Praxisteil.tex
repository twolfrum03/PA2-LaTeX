\chapter{Praktischer Teil}

-> Allgemeine Frage: Sind dir 3 Gliederungsebenen zu viel? Eine weitere Möglichkeit wäre ja, mehrere ''große'' Kapitel, anstatt nur Theorie- und Praxisteil

\section{Globaler Prozesskontext und Ist-Analyse}

-> Das Wording ''Globaler Prozesskontext'' gefällt mir noch nicht, mir ist bis jetzt noch nichts besseres eingefallen. Ich möchte hier kurz erläutern, wie mein betrachteter Prozess in den globalen Einkaufsprozess des Kunden einzuordnen ist, um dem Leser mehr Kontext zu geben

-> Ist-Analyse: Analysieren wie der Prozess aktuell aufgebaut ist und Verbesserungspotenziale bzw. Schwachstellen aufzeigen

\section{Anforderungsanalyse}

-> In dem Kapitel möchte ich im Gegensatz zum vorherigen eher auf Anforderungen eingehen, die seitens des Kunden an den neuen Prozess gestellt werden, aber auch Anforderungen aufnehmen, die sich aus dem vorherigen Kapitel ergeben haben

\section{Konzeption und Umsetzungsmöglichkeiten des optimierten Prozesses}

-> Hier geht es grundsätzlich darum, wie man den Prozess am besten anhand der Anforderungen optimiert und wie dieser in Zukunft ausgestaltet werden könnte 

-> Frage an dich: An sich gibt es nur zwei wirklich Lösungsansätze: Entweder man passt den Standard an (Customizing) oder man entscheidet sich für eine kundenspezifische Eigenentwicklung, reichen diese zwei Ansätze aus?

\subsection{Anpassung des Standards}

-> Hier möchte ich beschreiben, wie man den Prozess im System abbilden könnte, wenn man sich auf die Anpassung des Standards innerhalb der Customizing Grenzen beschränkt

-> Sollte es sinnvoll und auch für viele andere Kunden anwendbar sein, könnte auch in Abstimmung mit der SAP-Entwicklung der Standard erweitert/ angepasst werden (aber dann nicht mehr durch mich)

\subsection{Entwicklung einer ma\ss geschneiderten Lösung}

-> Dieser Lösungsansatz würde sich anbieten, wenn die Anforderungen des Kunden so speziell sind, dass sie nicht im Standard/ durch Customizing abgebildet werden können

-> Eine mögliche Ausgestaltung wäre die Entwicklung einer Fiori-App, über die durch API's die Daten nach der Vorstellung des Kunden im System gepflegt werden können, diese Lösung hätte aber einen enorm hohen Aufwand

\section{Evaluation der verschiedenen Lösungsansätze}

-> Hier sollen die in den vorherigen Kapiteln beschriebenen Ansätze evaluiert werden (Vor-, Nachteile, Kosten, Aufwand, Maintenance, etc.)

-> Die Evaluationskriterien ergeben sich dann aus den priorisierten/ gewichteten Anforderungen des Kunden