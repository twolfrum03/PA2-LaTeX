\chapter{Praktischer Teil \textcolor{red}{T-Shirt Size: XL}}

Im Folgenden werden die verschiedene praktische Lösungsansätze vorgestellt und anhand verschiedener Kriterien gegeneinander abgewogen, sodass am Ende eine Handlungsmatrix erstellt werden kann.

\section{Lösungsansätze}

Jetzt werden 3 verschiedene Lösungsansätze vorgestellt, wie man trotzdem sequentielle Prozesse/ asynchrone Kommunikation umsetzen kann

\subsection{Workflows \textcolor{red}{T-Shirt Size: L}}

Das ist die bisherige alte/ ineffiziente Lösung. Diese wird zwar noch vorgestellt, soll aber überdacht werden.

\subsection{Business Events \textcolor{red}{T-Shirt Size: L}}

Eine Option wären die Verwendung von Business Events. Hier auch ggf. auf Probleme mit Event-Mesh (Cloud- bzw. BTP-Komponente) für onPremise-Systeme eingehen -> lokale Verarbeitung der Business Events? 

\subsection{Background Processing Framework (bgpf) \textcolor{red}{T-Shirt Size: L}}

Andere Option wäre das Background Processing Framework über Background remote function calls.

\section{Entscheidungsmatrix \textcolor{red}{T-Shirt Size: M}}

Hier soll eine Entscheidungsmatrix entwickelt werden, welchen Lösungsansatz man in Abhängigkeit von mehreren Faktoren am besten verwenden soll (ersetzt auch weng mit die Zusammenfassung)