\chapter{Praktischer Teil}

Im Folgenden werden die verschiedene praktische Lösungsansätze vorgestellt und anhand verschiedener Kriterien gegeneinander abgewogen, sodass am Ende eine Handlungsmatrix erstellt werden kann.

\section{Lösungsansätze}

Zunächst sollen drei verschiedene Technologien vorgestellt werden, mit denen sich asynchrone Prozesse mit sequentieller Kommunikation, trotz den Einschränkungen durch RAP und Fiori Elements, umsetzen lassen.

\subsection{Business Workflows}

Der erste mögliche Ansatz sind Business Workflows (BW abgekürzt). BWs können benutzt werden um jeglichen Geschäftsprozess im SAP-System abzubilden. Sie decken das Spektrum von einfachen Genehmigungsprozessen bis hin zu komplexen Abläufen ab. Sie eignen sich vor allem für repetitive Prozesse mit mehreren Bearbeitern. BWs können zudem zur Fehelerbehandlung in anderen Prozessen oder eventgesteuert eingesetzt werden. Mit Workflows können durch die Benutzung der bereits bestehenden Funktionen und Transaktionen des SAP-Systems neue Geschäftsprozesse abgebildet werden. Die Funktionen und Transaktionen an sich werden dabei durch den Workflows nicht verändert. In Kombination mit Organisationsmanagement können die einzelnen Schritte des BW durch bestimmte Akteure ausgeführt werden. Das kann auch auf bestimmte Stellen abstrahiert werden, um von personellen Veränderungen innerhalb des Unternehmens unabhängig zu sein. Workflows können auch untereinander durch das Versenden und Konsumieren von Nachrichten kommunizieren. Diese Kommunikation ist auch zwischen verschiedenen SAP-Systemen über das Internet mit XML-Dokumenten möglich. \footcite[Vgl.][]{sap_business-workflows_2022-1}

Zunächst wird die Definition der Aufbau eines Business Workflows beschrieben. Diese lässt sich in vier Bereiche unterteilen.

\begin{figure}[H]
    \centering
    \includegraphics[height=8cm]{Bilder/Business-Workflows_Schema.png}
    \caption[Aufbau eines Business Workflows]{Aufbau eines Business Workflows, Abgerufen von \cite{sap_business-workflows_2022-1} am 17.07.2023.}
    \label{fig:iso_norm}
\end{figure}

Der Business Workplace ist der Ort in einem SAP-System, in dem der Endanwender ''work items'' (übersetzt aus dem Englischen: ''Arbeitspakete'' oder ''Aufgaben'') abhängig vom Zeitpunkt im Geschäftsprozess und den Berechtigungen des Users ausführen kann. Ein work item stellt zur Laufzeit des BW einen Schritt des Prozesses dar, der ausgeführt wird. Es werden hier jedoch nicht alle work items angezeigt. So werden \zB solche, die einem Prozess-Schritt, der im Hintergrund ausgeführt werden soll, zugeordnet sind, hier nicht angezeigt.

Ein Workflow muss vor Ausführung in der ''workflow definition'' angelegt werden. Diese Definition legt die Reihenfolge der auszuführenden Schritte des Prozesses fest und enthält zudem Kontrollschritte. Zusätzlich können noch Bearbeiter und Fristen für bestimmte Schritte festgelegt werden, die dann zur Laufzeit des Workflows vom ''Work item manager'' verwaltet werden. Es gibt viele Arten von Schritten, die gängigen Konzepten in der Programmierung ähneln, wie \zB normale Aktivitäten, Fallunterscheidungen, Schleifen. Zudem gibt es Schritte zum Versenden von Nachrichten, Auslösen von Events, Benutzerentscheidungen, usw. Diese Schritte können entweder im Dialog mit einem Benutzer ausgeführt werden, wenn \zB die Eingabe bestimmter Werte erforderlich ist, oder automatisch vom System im Hintergrund ausgeführt werden. Ein Workflow kann nicht nur manuell von einem Benutzer gestartet werden, sondern auch systemseitig von einem bestimmten Event ausgelöst werden. Hierfür muss in der Definition des Workflows das gewünschte Event als Auslöser angegeben werden. Wenn dann das Event auftritt, wird der Workflow automatisch gestartet. Im betrieblichen Kontext könnte hier \zB ein Mitarbeiter einen Urlaubsantrag stellen, der dann den als Workflow abgebildeten Genehmigungsprozess auslöst. Das wäre ein Beispiel für einen asynchronen Prozess mit sequentieller Kommunikation.

Die einzelnen Schritte, die innerhalb des Workflows ausgeführt werden, hei{\ss}en Tasks und stellen grundlegende betriebliche Tätigkeiten dar. Die Dialog- und Hintergrund-Schritte in der Workflow Definition korrespondieren hier mit Dialog- oder Hintergrund-Tasks. Im Workflow bezieht sich ein Task immer auf eine Methode eines Objekttyps. Diese Methoden können automatisch ausführbar sein oder müssen aktiv von einem Benutzer gestartet werden. Eine Methode kann einerseits Transaktionen oder Funktionen innerhalb des ERP-Systems aufrufen. Spezielle Anforderungen können durch kundeneigene Logik, oder Schnittstellen zu anderen Systemen umgesetzt werden.

Methoden, die innerhalb eines Workflows aufgerufen werden, sind immer Teil von Objekten. Diese Objekte können auch BOs sein. Im Allgemeinen ist ein Objekt ein konkreter Datensatz eines Objekttyps. Die Daten des Objekts werden durch seine Attribute definiert und die Aktionen, wie das Erstellen, Aktualisieren oder Löschen von Daten wird durch die Methoden des Objekts beschrieben. Einen weiteren wichtigen Teil von Objekten stellen Events dar. Diese werden ausgelöst, wenn bei einem Objekt seinen Status verändert. Das kann \zB durch das Erstellen, Verändern oder Löschen von Daten passieren. Diese Events können dann unter anderem Workflows starten. Das ''Business Object Repository'' bietet eine Übersicht über alle in einem SAP-System verfügbaren Objekttypen. Man kann die bereits vorhandenen Objekttypen bei Bedarf anpassen oder neue erstellen.

\subsection{Business Events}

Eine Option wären die Verwendung von Business Events. Hier auch ggf. auf Probleme mit Event-Mesh (Cloud- bzw. BTP-Komponente) für onPremise-Systeme eingehen -> lokale Verarbeitung der Business Events?

\subsection{Background Processing Framework}

Andere Option wäre das Background Processing Framework über Background remote function calls.

\section{Entscheidungsmatrix}

Hier soll eine Entscheidungsmatrix entwickelt werden, welchen Lösungsansatz man in Abhängigkeit von mehreren Faktoren am besten verwenden soll (ersetzt auch weng mit die Zusammenfassung)

Vergleichskriterien:

- BTP Event Mesh Cloud nötig in Systemlandschaft, andere Lösungen laufen nur lokal -> mehr Kosten,  Komplexität in Systemlandschaft, Datenschutz
- Kosten
- Performance (wshl verlieren BusinessEvents, Kommunikation über Systemgrenzen hinaus)
- Experteninterview: BusinessEvents (Martin Müller (+bgpf), Marcel Herrmanns, Daniel Wachs) 