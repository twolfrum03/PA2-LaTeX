\chapter{Theoretische Grundlagen}

-> Ich hätte generell die Frage, wie du die Unterkapitel im Bezug auf die anordnen würdest, also ob du die Reihenfolge sinnvoll findest und ob ich etwas vergessen habe

\section{SAP Central Procurement insb. Central Contracts}

->Allgemeine Vorstellung der Direct Sourcing Suite von SAP

-> Speziell auf Central Procurement eingehen, welchen Zweck erfüllt das Produkt, was sind wichtige Features, wer benutzt es?, ...

-> Spezieller Anwendungsfall Central Contracts: Was stellt das Objekt dar im Prozess, wie wird es genutzt, welche Daten werden dort abgelegt?, ...

\section{Geschäftsprozessanalyse und Prozessoptimierung}

-> Allgemeine Theorie zur Geschäftsprozessanalyse und Prozessoptimierung (wenn passende Literatur vorhanden auch direkt in Verbindung mit Massendaten-Management)

-> Welche Möglichkeiten gibt es Prozesse zu analysieren und wie lassen sich diese am besten optimieren?

-> Soll ich hier auf spezielle Frameworks etc. eingehen, bzw. was würde sich hier anbieten?

\section{(UI/ )UX im Geschäftsprozesskontext}

-> Ich würde UI fast weglassen, da sich mein Thema ja eher um die Optimierung des Prozesses dreht um die Benutzererfahrung zu verbessern <=> UI bezieht sich für mich eher auf die visuelle Gestaltung der Oberfläche, mit der ich mich ja eigentlich nicht auseinandersetzen wollte

-> Am besten wäre hier Literatur zu UX im Massendaten-Management-Kontext bzw. im Kontext von Business-Software, da das ja mein Anwendugsfall ist

\section{Massendaten-Management}

-> Hier möchte ich auch die allgemeine Theorie hinter einem effizienten Massendaten-Management erläutern, wie Daten möglichst effizient in Masse angelegt, verwaltet und geändert werden können