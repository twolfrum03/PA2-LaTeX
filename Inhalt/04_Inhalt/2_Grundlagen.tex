\chapter{Theoretische Grundlagen}

-> Nur Theorie, die später auch verwendet wird, nichts einfach so einführen

-> Voraussetzung: Basiswissen WI-Studium

\section{Geschäftsprozessanalyse und Prozessoptimierung}

-> Allgemeine Theorie zur Geschäftsprozessanalyse und Prozessoptimierung (wenn passende Literatur vorhanden auch direkt in Verbindung mit Massendaten-Management)

-> Darstellung von Methoden/ Frameworks zur Prozessanalyse, -optimierung

\section{User Experience im Geschäftsprozesskontext}

-> Literatur zu UX (allg., Massendaten-Management-Kontext, Business-Software-Kontext)

\section{Massendaten-Management}

-> allgemeine Theorie hinter effizientem Massendaten-Management erläutern (Anlage, Verwaltung, Änderung, Löschung)

\section{SAP Central Procurement insb. Central Contracts}

-> Direct Sourcing Suite von SAP

-> Central Procurement eingehen (Zweck des Produkts, wichtige Features, Anwender, ...)

-> Central Contracts (Was stellt das Objekt im Prozess dar, wie wird es genutzt, welche Daten werden dort abgelegt?, ...)