\chapter{Anhang}

\section{Fragebögen Experteninterviews}

\subsection{Workflows} \label{FB_Workflows}

\subsubsection{Allgemein}

Wer bist du und was ist deine Aufgabe in der AIS? Was machst du in deinem täglichen Alltag und wo kommst du mit Workflows in Kontakt?
Darf ich das Interview aufzeichnen, transkribieren und in meiner Praxisarbeit verwenden?

\subsubsection{Komplexität der Implementierung}

Wie komplex ist eine Implementierung von Workflows? (genauere Beschreibung der Implementierung, welche/ wie viele Artefakte müssen erstellt werden, ca. Angabe Lines of Code)

\subsubsection{Auswirkungen auf die Systemlandschaft}

Hat die Implementierung von Workflows Auswirkungen auf die Systemlandschaft? (Zusätzliche Systemkomponenten, Cloud (weitere Implikationen: Datenschutz, rechtliches, …) nötig, …)

\subsubsection{Performance}

Wie sind Workflows performance-technisch einzuordnen? (Rechenlast, Ausführungszeit, Netzwerklast) \newline
Ist für Workflows eine Kommunikation über Systemgrenzen hinaus notwendig?

\subsubsection{Kosten}

Entstehen durch die Implementierung von Workflows zusätzliche Kosten? (Lizenzkosten, Cloud-Abonnement (laufende Kosten), Netzwerk-Traffic)

\subsubsection{Flexibilität}

Wie flexibel sind Workflows im Bezug auf ihre spätere Anpassbarkeit, Gestaltungsmöglichkeiten bei speziellen Anforderungen, Integrationsmöglichkeiten mit anderen Technologien?

\subsubsection{Skalierbarkeit}

Wie skalierbar sind Workflows? (Aufteilbar (Load-Balancing) auf mehrere Systeme, Frameworks die Skalierung übernehmen) Bezug auf Fiori-Apps mit sehr hohem Nutzungsvolumen

\subsubsection{Wartbarkeit}

Beschreiben Sie die Wartbarkeit von Workflows (Wartung zentral  über mehrere Instanzen verteilt, Analysemöglichkeiten). 

\subsubsection{Abwärtskompatibilität}

Sind Workflows abwärtskompatibel zu älteren Releases bzw. älteren SAP-Systemen?

\subsection{Business Events} \label{FB_Business-Events}

\subsubsection{Allgemein}

Wer bist du und was machst du bei SAP? \newline
Wo kommst du mit Business Events in Kontakt? \newline
Darf ich das Interview aufzeichnen, transkribieren und in meiner Praxisarbeit verwenden?

\subsubsection{Komplexität der Implementierung}

Wie komplex ist eine Implementierung von Business Events? (genauere Beschreibung der Implementierung, welche/ wie viele Artefakte müssen erstellt werden)

\subsubsection{Auswirkungen auf die Systemlandschaft}

Hat die Implementierung von Business Events Auswirkungen auf die Systemlandschaft? (Zusätzliche Systemkomponenten, Umstrukturierung von Prozessen, Migrationsaufwand on-prem Landschaft zu eventgesteuerter Architektur, Cloud nötig, …)

\subsubsection{Performance}

Wie sind Business Events performance-technisch einzuordnen? (Rechenlast, Ausführungszeit, Netzwerklast) \newline
Ist für Business Events eine Kommunikation über Systemgrenzen hinaus notwendig?

\subsubsection{Kosten}

Entstehen durch die Implementierung von Business Events zusätzliche Kosten? (Lizenzkosten, Cloud-Abonnement, Netzwerk-Traffic) \newline
In welchem Verhältnis stehen diese zum Mehrwert, der durch eine eventgesteuerte Architektur entsteht?

\subsubsection{Flexibilität}

Wie flexibel sind Business Events im Bezug auf ihre spätere Anpassbarkeit, Gestaltungsmöglichkeiten bei speziellen Anforderungen, Integrationsmöglichkeiten mit anderen Technologien?

\subsubsection{Skalierbarkeit}

Wie skalierbar sind Business Events? (Aufteilbar (Load-Balancing) auf mehrere Systeme, Frameworks die Skalierung übernehmen) -> Bezug auf Fiori-Apps mit sehr hohem Nutzungsvolumen

\subsubsection{Wartbarkeit}

Beschreiben Sie die Wartbarkeit von Business Events (Wartung zentral  über mehrere Instanzen verteilt, Analysemöglichkeiten). 

\subsubsection{Abwärtskompatibilität}

Sind Business Events abwärtskompatibel zu älteren Releases bzw. älteren SAP-Systemen?

\subsection{bgPF} \label{FB_bgPF}

\section{Transkripte Expterteninterviews}

\subsection{Workflows} \label{T_Workflows}

\textbf{Befragender:} Tom Wolfrum (Abkürzung: \textbf{T})

\textbf{Befragter:} Eric Serie (Abkürzung: \textbf{E})

\textbf{Datum:} 21.07.2023

\begin{list}{X:}{\setlength{\labelsep}{5mm}}
    \linenumbers[1]
    \item[\textbf{T}:] Hallo Eric, vielen Dank, dass du dir die Zeit für dieses Interview im Rahmen meiner Praxisarbeit genommen hast. Thematisch soll es heute um die SAP-Technologie Business Workflows gehen. Doch starten wir bei dir als Person. Wer bist du und was ist deine Aufgabe in unserer Abteilung AIS HCM? Du kannst auch darauf eingehen, was du in deinem Arbeitsalltag machst und wo du mit Workflows in Kontakt kommst.
    \item[\textbf{E}:] Hallo Tom, ich bin Eric Serie. Ich habe 1997 bei SAP angefangen. Ich war anfangs in einer französischen Abteilung und jetzt seit über zehn Jahren in der AIS HCM. In meinem Arbeitsalltag kümmere ich mich größtenteils um Kundenmeldungen und helfe Kollegen bei Fragen. Mein Bereich ist die Personaladministration und ich betreue die Workflows der SAP Basis. Hier komme ich mit Workflows in Kontakt. Ich betreue den Teil der Workflows, die mit Personalentwicklung und Objekten wie Planstellen, Organisationseinheiten und Personalnummern zu tun haben.
    \item[\textbf{T}:] Danke für deine kurze Vorstellung. Noch vorneweg: Darf ich das Interview aufzeichnen, transkribieren und - wenn du damit einverstanden bist - in meiner Praxisarbeit verwenden? 
    \item[\textbf{E}:] Ja.
    \item[\textbf{T}:] Ok, super. Dann kommen wir zu den inhaltlichen Fragen. Zur Komplexität der Implementierung: Kannst du beschreiben, wie komplex eine Implementierung von Workflows ist, also was man genau machen muss und welche Artefakte erstellt werden müssen?
    \item[\textbf{E}:] Ein Workflow kann sehr einfach sein, also eine einfache Aufgabe erfüllen, wie \zB eine E-Mail versenden oder sehr komplex, da ein Workflow viele Aufgaben oder Schritte haben kann und diese können dann auch wieder sehr verschieden sein, wie zum Beispiel eine Entscheidung oder Genehmigung oder eine komplizierte Aufgabe, wie eine Abwesenheit anzulegen. Es gibt bei Workflows viele verschiedene Schritte und Schritttypen, deshalb können sie sehr komplex sein. Zudem kann in einem Schritt eines Workflows eigenes ABAP-Coding ausgeführt werden, was Workflows auch nochmal komplexer macht.
    \item[\textbf{T}:] Okay, dass heißt, dass die Komplexität eines Workflows sozusagen mit der Anzahl seiner Schritte steigt? 
    \item[\textbf{E}:] Ja, das ist auf jeden Fall so.
    \item[\textbf{T}:]  
\end{list}

\subsection{Business Events} \label{T_Business-Events}

\subsection{bgPF} \label{T_bgPF}