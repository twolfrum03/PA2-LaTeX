\chapter{Schlussbetrachtungen}

\section{Zusammenfassung}

Die vorliegende Arbeit befasst sich mit der Optimierung des Massenbearbeitungsprozesses von zentralen Einkaufskontrakten im Rahmen eines Beratungsprojekts bei einem Automobilhersteller. Der Fokus der Optimierung liegt in der Bearbeitung verschiedener Preisbestandteile, die insbesondere während der jährlichen Preisverhandlungen mit den Lieferanten von gro\ss er Bedeutung sind, da die neuen Preise in allen Kontrakten hinterlegt werden müssen. Zuerst wurden theoretische Grundlagen in den Bereichen Geschäftsprozessanalyse und Prozessoptimierung, User Experience und SAP-Produkte gelegt. Im Folgenden wurden nach einer Einordnung des Prozesses in den globalen S2P-Prozess von BMW die Anforderungen des Kunden ermittelt. Der Zielprozess soll hauptsächlich eine höhere Effizienz und eine verbesserte UX aufweisen. Diese wird durch die automatische Anpassung von Gültigkeitsintervallen der Preiskomponenten und die Vereinfachung der Bearbeitung durch verschiedene Vorauswahlschritte erreicht. Nach der Feststellung, dass SAP Ariba Central Procurement diese Anforderungen im Standard nicht erfüllen kann wurde ein Customizing des der Software innerhalb des Standards und die Entwicklung einer kundenspezifischen Applikation als mögliche Lösungen gegenübergestellt. Die beiden Möglichkeiten wurden anhand von Kriterien bewertet und auf dieser Basis die geeignetere Lösung ermittelt. Während die Anpassung des Standards durch Customizing und BAdIs gewisse Verbesserungen ermöglicht, bleibt die Prozessstruktur suboptimal. Eine kundenspezifische Fiori-App, basierend auf dem "Wizard-Floorplan", bietet hingegen eine bessere Usability, umfassende Anpassungsmöglichkeiten und erfüllt die Anforderungen vollständig. Die Evaluation durch eine Nutzwertanalyse zeigt, dass die kundenspezifische Lösung mit 20 von 30 Punkten besser abschneidet als die Standardlösung mit 17 Punkten.

\section{Fazit und Handlungsempfehlung}

% Die Anpassung des SAP-Standards, eignet sich aufgrund ihrer schnellen Implementierungszeit und niedrigeren initialen Kosten. Diese Methode hat jedoch den Nachteil einer suboptimalen Prozessstruktur und eingeschränkten Anpassungsmöglichkeiten. Der Entwicklungsaufwand ist gering, da lediglich ein BAdI implementiert werden muss und die Lösung ist in die bestehende Systemlandschaft integriert, jedoch bleibt die User Experience unzureichend und die Fehleranfälligkeit der Anwender hoch. 
% Im Gegensatz dazu bietet die Entwicklung einer kundenspezifischen Fiori-App, eine Prozessstruktur, die an den Arbeitsablauf der Einkäufer angepasst ist und somit eine sehr gute User Experience. Jedoch werden diese Vorteile durch erheblich höhere Entwicklungs- und Wartungskosten und eine längere Implementierungszeit begleitet. Diese Lösung bietet erhebliche Vorteile in der langfristigen Anpassbarkeit und Nützlichkeit, kann allerdings aufgrund der höheren Kosten und des größeren Schulungsaufwands zu Belastungen führen.

Die Anpassung des SAP-Standards, bietet den Vorteil einer schnellen Implementierung und geringeren initialen Kosten. Durch Customizing und  die Verwendung von BAdIs teilweise an die kundenspezifischen Anforderungen angepasst werden, was eine gewisse Flexibilität bietet. Zudem lässt sich die Standardlösung nahtlos in die bestehende Systemlandschaft integrieren, was die Komplexität der Implementierung vermindert. Jedoch zeigt sich, dass diese Methode in der Praxis durch strukturelle Einschränkungen und eine geringere Anpassungsfähigkeit an zukünftige Änderungen limitiert ist. Die Prozessstruktur bleibt suboptimal, da wesentliche Anforderungen wie die Vorauswahl von Zeiträumen und Kategorien nicht vollständig umgesetzt werden können. Dadurch werden Steigerungen in der Effizienz und Benutzerfreundlichkeit nur begrenzt erreicht. Im Gegensatz dazu bietet die Entwicklung einer kundenspezifischen Fiori-App, eine umfassendere Erfüllung der definierten Anforderungen. Diese Lösung ermöglicht eine vollständig angepasste Prozessstruktur, die den Arbeitsfluss der Endanwender berücksichtigt und umfasst Validierungs- und Simulationsmöglichkeiten. Diese methodische Benutzerführung führt zu einer besseren Benutzerakzeptanz und einer geringeren Fehleranfälligkeit. Zudem bietet die kundenspezifische Lösung langfristig eine höhere Flexibilität bei der Anpassung an sich verändernde Anforderungen und Prozesse. Allerdings ist dies mit höheren initialen Entwicklungs- und Wartungskosten sowie längeren Implementierungszeiten und Schulungsaufwänden verbunden. Trotz dieser Nachteile bietet die kundenspezifische Lösung eine nachhaltige Optimierung des Prozesses und stellt eine zukunftssichere Investition dar, die langfristige Verbesserungen bei der Effizienz und Benutzerfreundlichkeit ermöglicht.

Auf dieser Basis wird die Entwicklung einer BMW-spezifischen App empfohlen, da die kurzfristig geringeren Kosten nicht die langfristigen Produktivitätseinbu\ss en der Facheinkäufer aufwiegen. Somit wird sichergestellt, dass die Anwender Central Contracts effizient und fehlerfrei bearbeiten können und die Prozessqualität langfristig verbessert wird, wodurch die finanziellen Aufwände für die Entwicklung der App ausgeglichen werden sollten.

\section{Kritische Reflexion und Ausblick}

Zuletzt soll die vorliegende Arbeit noch kritisch reflektiert und ein Ausblick auf zukünftige Entwicklungen gegeben werden. In der Arbeit wurden zwei Möglichkeiten vorgestellt, wie der Massenbearbeitungsprozess von zentralen Einkaufskontrakten optimiert werden kann. Jedoch kann kritisiert werden, dass in der Betrachtung lediglich Umsetzungsmöglichkeiten innerhalb von SAP Produkten beleuchtet und Lösungen von anderen Anbietern au\ss er Acht gelassen wurden, die möglicherweise eine funktional bessere und kostengünstigere Alternative darstellen könnten. Dies ist dadurch bedingt, dass BMW die gesamte SAP Ariba Direct Material Sourcing Suite verwendet und auf strategischer Ebene entschieden wurde, dass die Systemlandschaft möglichst homogen gehalten werden soll. Des weiteren vereinfacht dies die Integration verschiedener Systeme erheblich, da \zB alle Masterdaten über standardisierte Schnittstellen bereits vorhanden sind. Insofern wäre die Betrachtung von Umsetzungsmöglichkeiten au\ss erhalb der SAP-Produkte für das Beratungsprojekt nicht zielführend gewesen.

Nachdem in dieser Arbeit eine geeignete Lösung identifiziert wurde und die grundlegende Konzeptionierung durchgeführt wurde wird im Projektkontext die Implementierung der kundenspezifischen Fiori-App erfolgen. Gegebenenfalls wird nach einer erfolgreichen Testphase bei BMW und der Evaluation durch das SAP Produktmanagement diese Lösung (in Teilen) in den SAP Standard übernommen. Auf akademischer Ebene ist, wie oben angesprochen, noch hohes Forschungspotential im Bezug auf die Umsetzung durch andere Lösungen vorhanden. Zudem ist das Gebiet der effizienten Massenbearbeitung im Kontext von Unternehmenssoftware noch weitesgehend unerforscht. Hier könnten weitere Arbeiten ansetzen, um allgemeine Methoden und Rahmenwerke für die effiziente Massenbearbeitung zu entwickeln.