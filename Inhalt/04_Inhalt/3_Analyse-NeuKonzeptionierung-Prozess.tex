\chapter{Analyse des Ist-Prozesses}

\section{Einordnung des betrachteten Prozesses}

Zunächst soll der in der vorliegenden Arbeit betrachtete Prozess in den Gesamtkontext des Beschaffungsprozesses des Kunden eingeordnet werden. Im Gesamtüberblick des Unternehmens befasst sich diese Arbeit mit dem ''Source-to-Pay''-Prozess (im Folgenden mit ''S2P'' abgekürzt). \footnote{Vgl. Anhang \ref{sec:AnhangA1}} S2P steht für den übergreifenden Beschaffungsprozess eines Unternehmens, der vom Entwickeln einer Beschaffungsstrategie über die Auswahl von Lieferanten und den Vertragsschluss bis hin zur Bestellung der Waren und Bezahlung des Lieferanten reicht. \footcite[Vgl.][S. 3]{praxis_jain_source_pay_definition_2017} Im Kundenkontext gliedert sich der S2P-Prozess in mehrere Unterprozesse auf: Einkauf direktes Material, Einkauf indirektes Material, M-Komponentenfertigung und Qualitätsmanagement-Teile. \footnote{Vgl. Anhang \ref{sec:AnhangA2}} Diese Arbeit befasst sich jedoch nur mit dem Prozess zur direkten Beschaffung, d.h. für Teile, die direkt der Leistungserstellung im Unternehmen dienen. Die Massenbearbeitung von Zentralkontrakten kommt hier hauptsächlich im Rahmen der Jahrespreisverhandlungen (im Schaubild \ref{sec:AnhangA3} als ''MatKo / JaVe'' bezeichnet). Einkäufer des Kunden verhandeln jährlich mit betreuten Lieferanten über Preise und Konditionen nach. Nachdem eine Einigung erzielt wurde, müssen die Central Contracts anhand der neuen Preise und Konditionen aktualisiert werden. 

\section{Anforderungsanalyse}

-> Ohne Betrachtung der Ist-Situation Anforderungen an einen neuen Prozess ermitteln (Greenfield-Ansatz)

\section{Bewertung der Ist-Situation}

-> Beurteilung des Ist-Stands anhand der im vorherigen Kapitel ermittelten Anforderungen

\section{Soll-Konzeption}

-> Allgemeine Konzeption eines optimierten Prozesses, der die Anforderungen erfüllt

-> Fokus setzen, was genau verbessert werden soll (alles meist unrealistisch)