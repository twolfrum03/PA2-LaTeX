\chapter{Analyse des Ist-Prozesses}

\section{Einordnung des betrachteten Prozesses}

Zunächst soll der in der vorliegenden Arbeit betrachtete Prozess in die Prozesslandschaft von BMW eingeordnet werden. Im Gesamtüberblick des Unternehmens befasst sich diese Arbeit mit dem ''Source-to-Pay''-Prozess (im Folgenden mit ''S2P'' abgekürzt).\footnote{Vgl. Anhang \ref{sec:AnhangA1}} S2P steht für den übergreifenden Beschaffungsprozess eines Unternehmens, der vom Entwickeln einer Beschaffungsstrategie über die Auswahl von Lieferanten und den Vertragsschluss bis hin zur Bestellung der Waren und Bezahlung des Lieferanten reicht. \footcite[Vgl.][S. 3]{praxis_jain_source_pay_definition_2017} Im Kundenkontext gliedert sich der S2P-Prozess in mehrere Unterprozesse auf: Einkauf direktes Material, Einkauf indirektes Material, M-Komponentenfertigung und Qualitätsmanagement-Teile. \footnote{Vgl. Anhang \ref{sec:AnhangA2}} Diese Arbeit befasst sich jedoch nur mit dem Prozess zur direkten Beschaffung, d.h. für Teile, die direkt der Leistungserstellung im Unternehmen dienen. Die Massenbearbeitung von Zentralkontrakten kommt hier hauptsächlich im Rahmen der Jahrespreisverhandlungen (im Schaubild \ref{sec:AnhangA3} als ''MatKo / JaVe'' bezeichnet). Einkäufer des Kunden verhandeln jährlich mit betreuten Lieferanten über Preise und Konditionen nach. Nachdem eine Einigung erzielt wurde, müssen die Central Contracts anhand der neuen Preise und Konditionen aktualisiert werden. An dieser Stelle wird die Massenbearbeitung von Zentralkontrakten benötigt, wofür im Folgenden ein Konzept entwickelt werden soll.

\section{Anforderungsanalyse}

Im Bereich der Massenbearbeitung von Zentralkontrakten wurden im Rahmen eines Experteninterviews mit dem Kunden drei Anforderungsbereiche identifiziert: allgemeine Anforderungen, Anforderungen wie sich das System bei bestimmten Eingaben verhalten soll und Anforderungen an die Struktur des Prozesses genannt.\footnote{Vgl. Anhang \ref{sec:AnhangA4}} Generell lässt sich feststellen, dass das Hauptaugenmerk technisch auf der Offline-Massenbearbeitung mit Excel liegt, da in jedem zu bearbeitenden Vertrag unterschiedliche Änderungen vorgenommen werden müssen. Fachlich sind im aktuellen Prozess vor allem die Bearbeitung von Basispreisen, Konditionen und Rohstoffen wichtig. Für Massenänderungen aller anderen einfachen Felder ist das die Online-Massenbearbeitung ausreichend. \footnote{Vgl. Anhang \ref{sec:AnhangA4}, Z. 52ff}

Eine allgemeine Anforderung ist, dass der Prozess effizienter und somit die Durchlaufzeit gesenkt wird, da ein Facheinkäufer sehr viele Verträge mit je unterschiedlichen Änderungen bearbeiten muss, da es finanziellen und rechtlichen Gründen wichtig ist, diese Änderungen möglichst schnell in den Systemen abzubilden.\footnote{Vgl. Anhang \ref{sec:AnhangA4}, Z. 36ff} Des Weiteren soll die User Experience im Allgemeinen verbessert werden. Hier soll der Fokus vor allem auf der einfachen Bedienbarkeit (Usability) durch den Endanwender gelegt werden, damit sich die Anwender möglichst schnell in der Lösung zurechtfinden und produktiv arbeiten können. Dennoch ist die Nützlichkeit des Systems auch sehr wichtig, da den Facheinkäufern die Arbeit erleichtert werden soll.\footnote{Vgl. Anhang \ref{sec:AnhangA4}, Z. 52ff}

Im Bereich der Anforderungen an das Systemverhalten wurde festgestellt, dass die Gültigkeitszeiträume der einzelnen Preiskomponenten sehr wichtig sind. Hier dürfen beim Einfügen neuer Intervalle die bestehenden Intervalle nicht gelöscht oder verschoben werden, wenn es zu einer Überlappung kommt. Stattdessen sollen diese automatisch angepasst werden, um Datenkonsistenz zu gewährleisten. Anpassung hei\ss t in diesem Fall, dass die vorhandenen Intervalle entweder anfangs oder am Ende so gekürzt werden, dass im Zeitstrahl eine Lücke entsteht, in die das neue Intervall ohne Überlappungen eingefügt werden kann.\footnote{Vgl. Anhang \ref{sec:AnhangA4}, Z. 77ff} Wenn ein neues Intervall in der Zukunft eingefügt wird, sodass zwischen dem einzufügenden und dem zeitlich vorherigen Intervall eine Lücke entstehen würde, soll das System automatisch, das letzte gültige Basispreisintervall bis zum Beginn des neuen Intervalls verlängern, da keine Zeiträume ohne gültige Preise existieren dürfen, da dies zu Fehlern in der Bestellung durch lokale Werkssysteme führen würde.\footnote{Vgl. Anhang \ref{sec:AnhangA4}, Z. 103ff} Ein Sonderfall sind Rohstoffkonditionen um die Preisentwicklung eines bestimmten Rohstoffs zu berücksichtigen. Diese Konditionen müssen aus bilanziellen Gründen immer eine Gültigkeit von einem Quartal haben, da sie quartalsweise berechnet werden. Beim Einfügen einer solchen Kondition muss das System automatisch das Intervall an den Quartalsgrenzen trennen und als neues Intervall fortführen, falls es sich über mehrere Quartale erstreckt.\footnote {Vgl. Anhang \ref{sec:AnhangA4}, Z. 89ff} Neben Funktionalitäten zur Gültigkeit einzelner Preisbestandteile soll das System zwar rückwirkende Änderungen ermöglichen, da die Jahrespreisverhandlungen in den meisten Fällen teilweise vergangene Zeiträume betreffen, jedoch nur innerhalb der vergangenen zwölf Monate. Für administrative Benutzer soll es möglich sein, rückwirkende Änderungen bis zu 36 Monaten vorzunehmen.\footnote{Vgl. Anhang \ref{sec:AnhangA4}, Z. 95ff} Zudem soll ein BMW-spezifisches Framework zur Prüfung der Preislogiken mit kundeneigenen Regeln in die Lösung eingebunden werden. Dieses soll die Eingaben des Einkäufers überprüfen und gegebenenfalls Warnungen oder Fehlermeldungen ausgeben, wenn die Eingaben nicht den Regeln entsprechen. \footnote{Vgl. Anhang \ref{sec:AnhangA4}, Z. 140ff}

Die Prozessstruktur betreffend soll die Massenpflege der Zentralkontrakte aufgegliedert werden. Hierbei soll in mehreren Schritten eine Vorauswahl stattfinden, um die auf einmal durch den Einkäufer zu berücksichtigenden Daten zu verringern und somit die Pflege der Verträge zu vereinfachen.\footnote{Vgl. Anhang \ref{sec:AnhangA4}, Z. 129ff} Nachdem alle Änderungen vorgenommen wurden, soll ein zusätzlicher Schritt zur Validierung und Simulation der Änderungen eingeführt werden, um Fehler zu vermeiden und dem Anwender die Möglichkeit zu geben, alle Änderungen auf deren Korrektheit zu überprüfen, bevor diese in das System übernommen werden.\footnote{Vgl. Anhang \ref{sec:AnhangA4}, Z. 138ff}

\section{Bewertung der Ist-Situation}

Aktuell verwendet BMW die Massenpflege von Central Contracts so, wie sie von SAP zur Verfügung gestellt wird, ohne spezifische Anpassungen, d.h. über die Fiori-App ''Massenänderungen an zentralen Einkaufskontrakten'', wie in Kapitel \ref{sec:Kapitel23MassChange} beschrieben. Zuerst lässt sich feststellen, dass diese für alle einfachen Felder aureichend ist, wenn beispielsweise Felder in meheren Kontrakten mit einem Wert überschrieben werden müssen. Der Standard bleibt jedoch hinter den Erwartungen zurück, wenn es um die Bearbeitung von Basispreisen, Konditionen und Rohstoffen geht, da hier spezielle Anforderungen von Kundenseite bestehen.

Im Bereich der allgemeinen Anforderungen ist der Prozess im Bezug auf Effizienz und Durchlaufzeit als nicht optimal zu bewerten. Dies resultiert aus dem komplexen Aufbau der Excel-Tabelle, wodurch die Anwender viel Zeit brauchen um die beabsichtigten Änderungen im System umzusetzen. Ein weiterer Nebeneffekt ist die hohe Fehlerquote, da die bestehende Lösung von den Einkäufern nicht verstanden wird. Die entstandenen Fehler müssen im Nachhinein durch einen Mitarbeiter des IT-Betriebs nachgebessert werden. Somit ist auch die User Experience als nicht gut zu bewerten.

Im Bezug auf das Systemverhalten erfüllt die Lösung von SAP im Standard schon die Anforderungen, dass beim Einfügen eines Basispreis-Intervalls die bestehenden Intervalle nicht gelöscht oder verschoben, sondern automatisch gekürzt werden. Die Anforderung, dass beim Einfügen eines Intervalls in der Zukunft das letzte gültige Basispreisintervall bis zum Beginn des neuen Intervalls verlängert wird, damit keine Lücke ohne gültigen Preis entsteht, wird jedoch nicht erfüllt. Aktuell wird die entstehende Lücke von System ohne weiteres akzeptiert. Auch die automatische Trennung von Rohstoffkonditionen an den Quartalsgrenzen wird nicht unterstützt. Das System erlaubt das Einfügen von beispielsweise zwölf-monatigen Intervallen bei Rohstoffen, ohne dass diese an den Quartalsgrenzen getrennt werden. Der Zeitraum, in dem Änderungen rückwirkend möglich sind, ist ebenfalls unbeschränkt. Somit werden sowohl die zwölf-, als auch die 36-Monatsgrenze überschritten. Des weiteren findet keine Unterscheidung zwischen administrativen und normalen Benutzern statt. Das BMW-spezifische Framework zur Prüfung von Preislogiken ist nicht im Standard eingebunden und die Regeln finden somit auch keine Anwendung.

Die Prozessstruktur betreffend ist eine Aufgliederung der Massenpflege der Zentralkontrakte in mehrere Schritte im Standard nicht vorhanden. Es können lediglich Kontrakte bzw. Kontraktpositionen selektiert und deren gesamte Daten heruntergeladen werden. Eine Einschränkung auf \zB einen gewissen Zeitraum ist nicht möglich. Die Simulation einer Massenänderung ist hingegen schon vorhanden. In dieser werden dem Facheinkäufer auch Warnungen und Fehlermeldungen angezeigt. Dennoch sind letztere für einen nicht technisch versierten Anwender schwer verständlich und der Anwender erfährt nicht, an welcher Stelle der Tabelle er einen Fehler gemacht hat und worin genau dieser besteht.  Eine Übersichtsseite, auf der der Facheinkäufer alle Änderungen an den verschiedenen Verträgen auf einen Blick sehen kann, existiert nicht.

\section{Soll-Konzeption}

Nach der Analyse der Anforderungen von BMW und der Bewertung der Ist-Situation soll anschlie\ss end ein allgmeines Konzept für einen optimierten Prozess entwickelt werden, der möglichst viele Anforderungen erfüllt und aktuelle Schwächen behebt. Dieses Konzept soll zunächst unabhängig vom bestehenden Standard bzw. Customizing und eventuellen Möglichkeiten einer Eigenentwicklung erarbeitet werden.

\begin{figure}[H]
    \centering
    \includegraphics[height=0.7cm]{Bilder/Praxisteil-Konzept-Prozess.png}
    \caption[Konzeptuelle Prozessstruktur der Massenbearbeitung]{Konzeptuelle Prozessstruktur der Massenbearbeitung. Eigene Darstellung}
    \label{fig:PraxisKonzeptProzess}
\end{figure}

Um den neuen Prozess aufzubauen wird zuerst dessen Struktur in Abbildung \ref{fig:PraxisKonzeptProzess} festgelegt. Eine zentrale Anforderung des Kunden ist, die Massenpflege durch mehrere Vorauswahl-Schritte zu vereinfachen. Deshalb soll als Erstes das zu bearbeitende Zeitintervall eingegrenzt werden, damit der Einkäufer während der Änderungsphase die Gültigkeiten seiner Änderungen nicht mehr berücksichtigen muss Hierfür muss der Facheinkäufer einen Datumsbereich auswählen können. Im nächsten Schritt sollen die Central Contracts, die angepasst werden sollen, selektiert werden. Um diesen Schritt zu vereinfachen wird eine Filter- und Suchfunktionalität benötigt. Nachdem die Verträge und das Zeitintervall festgelegt wurden ist noch der Umfang der Änderungen zu bestimmen. Konkret gibt es fünf Kategorien: Den Basispreis, Zu- und Abschläge, Fremdwährungen, marktorientierte Rohstoffe (im Folgenden ''RMO'' abgekürzt) und Rohstoffe mit freier Notierung (im Folgenden mit ''RIK'' abgekürzt). Diese Kategorien wurden als relevant für die Massenänderung identifiziert, da sich der letztendliche Bestellpreis aus ihnen zusammensetzt und alle Preisbestandteile in Masse änderbar sein müssen. Letztere müssen vom Anwender zur Änderung ausgewählt und innerhalb dieser konkrete Konditionen oder Rohstoffe selektiert werden können. Somit kann vorgebeugt werden, dass der Facheinkäufer nur beabsichtigte Änderungen durchführt. In der nächsten Phase des Prozesses werden die Änderungen im Rahmen der Vorauswahl durchgeführt. Da der Basispreis die Grundlage der Preisfindung bildet, soll dieser als Erstes angepasst werden. Wichtige Felder sind hierbei Betrag, Währung und Preis-Mengen-Einheit (Im Folgenden ''PME'' abgekürzt). Die PME setzt sich aus der Stückzahl und deren Messgrö\ss e zusammen und beschreibt pro welchem Betrag mit welcher Messgrö\ss e der Preis gilt. Beispielsweise kann ein Preis von 10\euro\ pro 100 Stück oder pro 5 m$^2$ gelten. Des Weiteren kann ein Basispreis, sowie jede der nachfolgenden Kategorien eine werksspezifische Gültigkeit haben, wodurch globale Preisdifferenzen abgebildet werden können. Letztere können global für alle Werke oder nur lokal für ein einzelnes Werk gültig sein. Nachdem die Basispreise geändert wurden, werden im nächsten Schritt Zu- und Abschläge bearbeitet. Da diese prozentual auf den Basispreis oder absolute Werte sein können, ist hier neben dem Betrag, der Währung und Werksabhängigkeit wichtig, ob es sich um einen prozentualen Zuschlag handelt. Fremdwährungsanteile sind im sechsten Prozessschritt die dritte Kategorie, die verändert wird. Diese sind im Bezug auf die zu bearbeitenden Felder übereinstimmend mit den Zu-/ Abschlägen. Die vierte Gruppe sind RMO. Die ausschlaggebenden Felder unterscheiden sich zu den vorherigen Kategorien: Neben dem Gewicht, welches sich aus dem Betrag und der Messgrö\ss e zusammensetzt ist die Beteiligungsquote und die Notation, die sich auch Betrag und Währung, sowie PME aufbaut, wichtig. Daneben ist ebenfalls die Werksabhängigkeit des Rohstoffs zu beachten. Die Beteiligungsquote gibt an, zu welchem Anteil sich BMW an den Fluktuationen des Rohstoffpreises beteiligt. Die PME bildet im Rohstoffkontext ab, pro welchem Gewicht der Preis gilt. Übliche Werte sind hier pro Tonne, Kilogramm oder Gramm. Da die Preise von RMO an der Börse bestimmt werden, soll die PME nicht durch den Facheinkäufer modifizierbar sein. Der letzte Bereich sind RIK. Diese gleichen den RMO, bis auf dass deren Notation durch den Endanwender frei anpassbar sein soll. Im letzten Schritt des Prozesses soll es dem Einkäufer möglich sein, seine in den vorherigen Schritten getätigten Änderungen zu überprüfen. Hierfür müssen diese inklusive der simulierten Auswirkungen in einer Übersicht dargestellt werden. Nach der Überprüfung kann der Nutzer entscheiden, ob die Änderungen in das System übernommen werden sollen oder nicht.

Nachdem der Prozessablauf definiert wurde, soll nun das Systemverhalten beschrieben werden: Da der Facheinkäufer sich in der Vorauswahl neben den Zentralkontrakten auch auf ein Zeitintervall festgelegt hat, sollen alle bestehenden Daten, mit den vorgenommenen Änderungen, im Rahmen dieses Intervalls überschrieben werden. Hierbei ist zu beachten, dass die bestehenden Intervalle nicht gelöscht oder verschoben, sondern automatisch so gekürzt werden sollen, dass das neue Intervall ohne Überlappungen oder Lücken eingefügt werden kann. Wird ein neues Intervall in der Zukunft eingefügt, soll das Sytem das letzte gültige Basispreisintervall bis zum Beginn des neuen Intervalls automatisch verlängern, damit kein Zeitraum ohne gültigen Preis entsteht. Eine weitere automatische Anpassung muss bei Rohstoffen erfolgen. Diese müssen durch das System automatisch an Quartalsgrenzen getrennt werden, sollte sich das vom Anwender gewählte Intervall über Quartalsgrenzen hinausgehen. Der Zeitraum, in dem Änderungen durchgeführt werden, muss anhand unterschiedlicher Berechtigungen für verschiedene Usergruppen eingeschränkt werden. Facheinkäufer sollen die letzten zwölf Monate ändern können, während bei administrativen Benutzern 36 Monate möglich sein sollen. Das BMW-spezifische Framework zur Prüfung von Preislogiken soll ab dem ersten Bearbeitungsschritt in die Lösung eingebunden werden, um den Endanwender direkt bei der Eingabe auf fehlerhafte Werte aufmerksam zu machen, damit dieser den Prozess nicht auf Basis falscher Annahmen fortsetzen kann.

Durch die Einführung einer neuen Prozessstruktur in Verbindung mit dem beschriebenen Systemverhalten soll der Prozess optimiert werden. Das Ziel ist es, die Durchlaufzeit und Fehlerquote zu senken, sowie die User Experience zu verbessern. Die funktionalen Anforderungen müssen unbedingt erfüllt werden. Daneben soll eine möglichst gute UX erreicht werden.