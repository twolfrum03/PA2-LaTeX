% ------------------------------------------------------------

% LaTeX Template für die DHBW zum Schnellstart!
% Original: https://github.wdf.sap.corp/vtgermany/LaTeX-Template-DHBW
% ------------------------------------------------------------
% ---- Präambel mit Angaben zum Dokument
\documentclass[
	fontsize=12pt,           % Leitlinien sprechen von Schriftgröße 12.
	paper=A4,
	twoside=false,
	listof=totoc,            % Tabellen- und Abbildungsverzeichnis ins Inhaltsverzeichnis
	bibliography=totoc,      % Literaturverzeichnis ins Inhaltsverzeichnis aufnehmen
	titlepage,               % Titlepage-Umgebung anstatt \maketitle
	headsepline,             % horizontale Linie unter Kolumnentitel
	abstract,              % Überschrift einschalten, Abstract muss in {abstract}-Umgebung stehen
]{scrreprt}                  % Verwendung von KOMA-Report
\usepackage[utf8]{inputenc}  % UTF8 Encoding einschalten
\usepackage[ngerman]{babel}  % Neue deutsche Rechtschreibung
\usepackage[T1]{fontenc}     % Ausgabe von westeuropäischen Zeichen (auch Umlaute)
\usepackage{microtype}       % Trennung von Wörtern wird besser umgesetzt
\usepackage{lmodern}         % Nicht-gerasterte Schriftarten (bei MikTeX erforderlich)
\usepackage{graphicx}        % Einbinden von Grafiken erlauben
\usepackage{wrapfig}         % Grafiken fließend im Text
\usepackage{setspace}        % Zeilenabstand \singlespacing, \onehalfspaceing, \doublespacing
% \usepackage[
% 	%showframe,                % Ränder anzeigen lassen
% 	left=3.5cm, right=2.5cm,
% 	top=1.25cm,  bottom=0.75cm,
% 	includeheadfoot, headsep=1.25cm, footskip=1.25cm
% ]{geometry}                      % Seitenlayout einstellen
\usepackage[left=3.5cm, right=2.5cm, head=1.25cm, bottom=2cm, foot=1.25cm, includefoot]{geometry}

\usepackage{scrlayer-scrpage}    % Gestaltung von Fuß- und Kopfzeilen
\usepackage{acronym}             % Abkürzungen, Abkürzungsverzeichnis
\usepackage{titletoc}            % Anpassungen am Inhaltsverzeichnis
\contentsmargin{0.75cm}          % Abstand im Inhaltsverzeichnis zw. Punkt und Seitenzahl
\usepackage[                     % Klickbare Links (enth. auch "nameref", "url" Package)
  hidelinks,                     % Blende die "URL Boxen" aus.
  breaklinks=true                % Breche zu lange URLs am Zeilenende um
]{hyperref}
\usepackage[hypcap=true]{caption}% Anker Anpassung für Referenzen

\usepackage{lineno}

\usepackage{chngcntr}
\counterwithout{figure}{chapter}

\renewcommand{\thefigure}{\arabic{figure}}
\renewcommand{\thetable}{\arabic{table}}

\urlstyle{same}                  % Aktuelle Schrift auch für URLs
% Anpassung von autoref für Gleichungen (ergänzt runde Klammern) und Algorithm.
% Anstatt "Listing" kann auch z.B. "Code-Ausschnitt" verwendet werden. Dies sollte
% jedoch synchron gehalten werden mit \lstlistingname (siehe weiter unten).
\addto\extrasngerman{%
	\def\equationautorefname~#1\null{Gleichung~(#1)\null}
	\def\lstnumberautorefname{Zeile}
	\def\lstlistingautorefname{Listing}
	\def\algorithmautorefname{Algorithmus}
	% Damit einheitlich "Abschnitt 1.2[.3]" verwendet wird und nicht "Unterabschnitt 1.2.3"
	% \def\subsectionautorefname{Abschnitt}
}

% ---- Abstand verkleinern von der Überschrift 
\renewcommand*{\chapterheadstartvskip}{\vspace*{.5\baselineskip}}

% Hierdurch werden Schusterjungen und Hurenkinder vermieden, d.h. einzelne Wörter
% auf der nächsten Seite oder in einer einzigen Zeile.
% LaTeX kann diese dennoch erzeugen, falls das Layout ansonsten nicht umsetzbar ist.
% Diese Werte sind aber gute Startwerte.
\widowpenalty10000
\clubpenalty10000

% ---- Für das Quellenverzeichnis
\usepackage[
	backend = biber,                % Verweis auf biber
	language = auto,
	style = numeric,                % Nummerierung der Quellen mit Zahle
	citestyle=authoryear,
	sorting = none,                 % none = Sortierung nach der Erscheinung im Dokument
	sortcites = true,               % Sortiert die Quellen innerhalb eines cite-Befehls
	block = space,                  % Extra Leerzeichen zwischen Blocks
	hyperref = true,                % Links sind klickbar auch in der Quelle
	backref = true,                % Referenz, auf den Text an die zitierte Stelle
	bibencoding = auto,
	giveninits = true,              % Vornamen werden abgekürzt
	doi=false,                      % DOI nicht anzeigen
	isbn=false,                     % ISBN nicht anzeigen
    alldates=short                  % Datum immer als DD.MM.YYYY anzeigen
]{biblatex}
\addbibresource{Inhalt/literatur.bib}
\setcounter{biburlnumpenalty}{3000}     % Umbruchgrenze für Zahlen
\setcounter{biburlucpenalty}{6000}      % Umbruchgrenze für Großbuchstaben
\setcounter{biburllcpenalty}{9000}      % Umbruchgrenze für Kleinbuchstaben
\DeclareNameAlias{default}{family-given}  % Nachname vor dem Vornamen
\AtBeginBibliography{\renewcommand{\multinamedelim}{\addslash\space
}\renewcommand{\finalnamedelim}{\multinamedelim}}  % Schrägstrich zwischen den Autorennamen
\DefineBibliographyStrings{german}{
  urlseen = {Einsichtnahme:},                      % Ändern des Titels von "besucht am"
}
\usepackage[babel,german=quotes]{csquotes}         % Deutsche Anführungszeichen + Zitate


\usepackage{xcolor}
\usepackage{blindtext}
\usepackage{soul}

% ---- Für Mathevorlage
\usepackage{amsmath}    % Erweiterung vom Mathe-Satz
\usepackage{amssymb}    % Lädt amsfonts und weitere Symbole
\usepackage{MnSymbol}   % Für Symbole, die in amssymb nicht enthalten sind.


% ---- Für Quellcodevorlage
\usepackage{scrhack}                    % Hack zur Verw. von listings in KOMA-Script
\usepackage{listings}                   % Darstellung von Quellcode
\usepackage{xcolor}                     % Einfache Verwendung von Farben
\input{Inhalt/00_Latex/quellcodeStyle}  % Weitere Details sind ausgelagert

\usepackage{algorithm}                  % Für Algorithmen-Umgebung (ähnlich wie lstlistings Umgebung)
\usepackage{algpseudocode}              % Für Pseudocode. Füge "[noend]" hinzu, wenn du kein "endif",
                                        % etc. haben willst.

\makeatletter                           % Sorgt dafür, dass man @ in Namen verwenden kann.
                                        % Ansonsten gibt es in der nächsten Zeile einen Compilefehler.
\renewcommand{\ALG@name}{Algorithmus}   % Umbenennen von "Algorithm" im Header der Listings.
\makeatother                            % Zeichen wieder zurücksetzen
\renewcommand{\lstlistingname}{Listing} % Erlaubt das Umbenennen von "Listing" in anderen Titel.


% ---- Tabellen
\usepackage{booktabs}  % Für schönere Tabellen. Enthält neue Befehle wie \midrule
\usepackage{multirow}  % Mehrzeilige Tabellen
\usepackage{siunitx}   % Für SI Einheiten und das Ausrichten Nachkommastellen
\sisetup{locale=DE, range-phrase={~bis~}, output-decimal-marker={,}} % Damit ein Komma und kein Punkt verwendet wird.
\usepackage{xfrac} % Für siunitx Option "fraction-function=\sfrac"

% ---- Für Definitionsboxen in der Einleitung
\usepackage{amsthm}                     % Liefert die Grundlagen für Theoreme
\usepackage[framemethod=tikz]{mdframed} % Boxen für die Umrandung
\input{Inhalt/00_Latex/highlightBoxen}  % Weitere Details sind ausgelagert

% ---- Für Todo Notes
\usepackage{todonotes}
\setlength {\marginparwidth }{2cm}      % Abstand für Todo Notizen

\usepackage[official]{eurosym}

\usepackage{pdfpages}

\usepackage[ngerman]{babel}

% ---- Elektronische Version oder Gedruckte Version?
% ---- Unterschied: Die elektronische Version enthält keinen Platzhalter für die Unterschrift
\usepackage{ifthen}
\usepackage{color}
\newboolean{e-Abgabe}
\setboolean{e-Abgabe}{false}    % false=gedruckte Fassung

% ---- Persönlichen Daten:
\newcommand{\titel}{Optimierung eines Massendatenimports in S/4 HANA Central Procurement am Beispiel eines Prozesses in der Automobilbranche}
\newcommand{\titelheader}{Prozessoptimierung eines Massendatenimports}
\newcommand{\arbeit}{Projektarbeit 2}
\newcommand{\studiengang}{Wirtschaftsinformatik}
\newcommand{\studienjahr}{2024}
\newcommand{\autor}{Tom Wolfrum}
\newcommand{\autorReverse}{Wolfrum, Tom}
\newcommand{\verfassungsort}{Karlsruhe}
\newcommand{\matrikelnr}{4000776}
\newcommand{\kurs}{WWI22B5}
% \newcommand{\bearbeitungsmonat}{Januar 2018}
\newcommand{\abgabe}{4. September 2023}
\newcommand{\bearbeitungszeitraum}{29.04.2024 - 02.09.2024}
\newcommand{\firmaName}{SAP SE}
\newcommand{\firmaStrasse}{Dietmar-Hopp-Allee 16}
\newcommand{\firmaPlz}{69190 Walldorf, Deutschland}
\newcommand{\betreuerFirma}{Steven Rösinger}
\newcommand{\betreuerDhbw}{Pascal Klimek}

\input{Inhalt/00_Latex/kopfundFusszeile}

% ---- Hilfreiches
\newcommand{\zB}{z.\,B. }   % "z.B." mit kleinem Leeraum dazwischen (ohne wäre nicht korrekt)
\newcommand{\dash}{d.\,h. }

\newcommand{\code}[1]{\texttt{#1}} % Ist einfacher zu schreiben als ständig \texttt und erlaubt
                                   % Änderungen im Nachhinein, wenn man z.B. Inline-Code anders stylen möchte.

% ---- Silbentrennung (falls LaTeX defaults falsch / nicht gewünscht sind)
\hyphenation{HANA}         % anstatt HA-NA
\hyphenation{Graph-Script} % anstatt GraphS-cript

% ---- Beginn des Dokuments

\begin{document}
\setlength{\parindent}{0pt}              % Keine Paragraphen Einrückung.
                                         % Dafür haben wir den Abstand zwischen den Paragraphen.
\setcounter{secnumdepth}{2}              % Nummerierungstiefe fürs Inhaltsverzeichnis
\setcounter{tocdepth}{2}                 % Tiefe des Inhaltsverzeichnisses. Ggf. so anpassen,
                                         % dass das Verzeichnis auf eine Seite passt.
\sffamily                                % Serifenlose Schrift verwenden.

% ------ Vorspann
% ------ Titelseite
\singlespacing
\thispagestyle{empty}
\begin{titlepage}
\enlargethispage{4cm}

\begin{figure}           % Logo vom Ausbildungsbetrieb und der DHBW
	\vspace*{-5mm}       % Sollte dein Titel zu lang werden, kannst du mit diesem "Hack" 
	%                      den Inhalt der Seite nach oben schieben.
	\begin{minipage}{0.49\textwidth}
		\flushleft
		\includegraphics[height=2.5cm]{Bilder/Logos/Logo_SAP.pdf} 
	\end{minipage}
	\hfill
	\begin{minipage}{0.49\textwidth}
		\flushright
		\includegraphics[height=2.5cm]{Bilder/Logos/Logo_DHBW.pdf} 
	\end{minipage}
\end{figure} 
\vspace*{0.1cm}

\begin{center}
	\begin{spacing}{0.9}
		\huge{\textbf{\titel}}\\[1.5cm]
	\end{spacing}
	\Large{\textbf{\arbeit}}\\[0.5cm]
	\normalsize{im Rahmen der Prüfung zum\\[1ex] \textbf{Bachelor of Science (B.Sc.)}}\\[0.5cm]
	\Large{des Studienganges \studiengang}\\[1ex]
	\normalsize{an der Dualen Hochschule Baden-Württemberg Karlsruhe}\\[1cm]
	\normalsize{von}\\[1ex] \Large{\textbf{\autor}} \\[1cm]

	% Hinweis: Manche Dozenten möchten einen Hinweis auf den Sperrvermerk auf der Titelseite.

	% Sperrvermerkt ein-/auskommentieren:
	\large{{\color{red}- Sperrvermerk -}}\\[1cm]


\end{center}

\begin{center}
	\vfill
	\begin{tabular}{ll}
		Abgabedatum:                     & \abgabe \\[0.2cm]
		Bearbeitungszeitraum:            & \bearbeitungszeitraum \\[0.2cm]
		Kurs:            				 & \kurs \\[0.2cm]
		Ausbildungsfirma:                & \firmaName \\
		                                 & \firmaStrasse \\
		                                 & \firmaPlz \\[0.2cm]
		Betreuer der Ausbildungsfirma:   & \betreuerFirma \\[0.2cm]
		Gutachter der Dualen Hochschule: & \betreuerDhbw \\[2cm]
	\end{tabular} 
\end{center}
\end{titlepage}
  % Titelseite
\newcounter{savepage}
\pagenumbering{Roman}                    % Römische Seitenzahlen
\onehalfspacing

% ------ Erklärung, Sperrvermerk, Abstact
\include{Inhalt/01_Standard/sperrvermerk}
\include{Inhalt/01_Standard/erklaerung}
\include{Inhalt/01_Standard/geschlechtsneutral}

%\include{Inhalt/02_Abstract/abstract-en}y
%\include{Inhalt/02_Abstract/abstract-de}

% ------ Inhaltsverzeichnis
\singlespacing
\small
\tableofcontents
\normalsize

% ------ Verzeichnisse
\renewcommand*{\chapterpagestyle}{plain}
\pagestyle{plain}
%\include{Inhalt/03_Verzeichnisse/formelgroessen}
\chapter*{Abkürzungsverzeichnis}
\addcontentsline{toc}{chapter}{Abkürzungsverzeichnis} % Hinzufügen zum Inhaltsverzeichnis 

\begin{acronym}[WYSISWG] % längstes Kürzel wird verw. für den Abstand zw. Kürzel u. Text

	% Alphabetisch selbst sortieren - nicht verwendete Kürzel rausnehmen!
	
	% Bsp.:
	\acro{DMS}{SAP Ariba Direct Materials Sourcing for Automotive and Industrial Manufacturing in SAP S/4 HANA}
	\acro{S/4}{S/4 HANA}
	\acro{PLM}{Product Lifecycle Management}
	\acro{CP}{SAP Ariba Central Procurement}
	\acro{GP}{Geschäftsprozess}
	\acro{GPA}{Geschäftsprozessanalyse}
	\acro{GPO}{Geschäftsprozessoptimierung}

\end{acronym}
\listoffigures                          % Erzeugen des Abbildungsverzeichnisses 
\listoftables                           % Erzeugen des Tabellenverzeichnisses
\renewcommand{\lstlistlistingname}{Quellcodeverzeichnis}
%\lstlistoflistings                      % Erzeugen des Listenverzeichnisses
\setcounter{savepage}{\value{page}}


% ------ Inhalt der Arbeit
\cleardoublepage
\pagenumbering{arabic}                  % Arabische Seitenzahlen für den Hauptteil
\setlength{\parskip}{0.5\baselineskip}  % Abstand zwischen Absätzen
\rmfamily
\renewcommand*{\chapterpagestyle}{scrheadings}
\pagestyle{scrheadings}
\onehalfspacing
%\include{Inhalt/04_Inhalt/einleitung}
%\include{Inhalt/04_Inhalt/formatText}
%\include{Inhalt/04_Inhalt/abbildungen}
%\include{Inhalt/04_Inhalt/mathematische-formeln}
%\include{Inhalt/04_Inhalt/quellcode}
%\include{Inhalt/04_Inhalt/literaturHinweis}

% \include{Inhalt/04_Inhalt/einleitung.tex}
% \include{Inhalt/04_Inhalt/grundlagen.tex}
% \include{Inhalt/04_Inhalt/dynamischeLearningNFTs.tex}
% \include{Inhalt/04_Inhalt/anwendbarkeitfürXGP.tex}
% \include{Inhalt/04_Inhalt/schlussbetrachtung.tex}

\chapter{Einleitung}

%Umfang: ca. 2-3 Seiten

\section{Motivation und Problemstellung}

Durch die Digitalisierung, zunehmende Komplexität globaler Lieferketten, starken Preisdruck der Konkurrenz und dem Wechsel zur nachhaltigen Mobilität befindet sich die Automobilbranche in einem bedeutenden Wandel. Der Einkauf ist seither ein gro\ss er Hebel, um die Produktionskosten zu senken und dadurch die Margen erhöhen zu können. Deshalb kommt einem optimalen Beschaffungssystem eine immer grö\ss ere Bedeutung zu. Ein wichtiger Bestandteil ist hierbei ein effizientes Massendaten-Management, da Einkäufer vor der Herausforderung stehen, die Datenmengen, die mit der komplexen Beschaffung vieler Teile einhergehen, zu bewältigen. 

Im Kontext des ''connected Procurement''-Beratungsprojekts möchte der deutsche Automobilhersteller BMW die SAP Produktsuite für die direkte Materialbeschaffung auf Basis von S/4 HANA einführen und somit Einkaufsprozesse digitalisieren und optimieren. Unter anderem soll die Zusammenarbeit mit Lieferanten in zentralen Einkaufskontrakten verwaltet werden. Die Preisbestandteile einzelner zu beschaffender  Bauteile müssen in diesen Central Contracts jährlich im Rahmen der Preisverhandlungen aktualisiert werden. Dafür wird eine Möglichkeit, um die Verträge effizient in Masse zu bearbeiten benötigt, da die Benutzung der SAP-Standardfunktionalität aufgrund von hohem Zeitaufwand und Fehleranfälligkeit nicht infrage kommt. Aufgrund der strategischen Relevanz des Kunden und dessen hoher Priorisierung des Prozesses wird zudem eine Übernahme der entwickelten Lösung in den SAP-Standard in Betracht gezogen.

\section{Ziel der Arbeit}

Das Ziel dieser Arbeit ist es, eine Handlungsempfehlung für den Kunden zu auszusprechen, wie der Massenbearbeitungsprozess von Zentralkontrakten innerhalb des SAP-Produkts optimiert werden kann. Es soll die Frage beantwortet werden, wie der Prozess am Besten gestaltet und umgesetzt werden kann, um für den Endanwender eine effiziente Bearbeitung der Verträge mit einer hohen Benutzerfreundlichkeit zu ermöglichen. Dies soll durch die Analyse des Ist-Zustands und der anschlie\ss enden Konzeption zweier Lösungsmöglichkeiten anhand der Anforderungen des Kunden ermöglicht werden. Durch die Bewertung beider Lösungen sollen die Vor- und Nachteile letzterer herausgearbeitet und dadurch eine Empfehlung für die Abbildung des Prozesses gegeben werden.  

% -> Wichtigster Teil der Einleitung (Ziel der Arbeit in 1. Satz auf den Punkt bringen, danach mehr ausführen, hier Forschungsfrage rein)
% -> Das Ziel der Arbeit muss bei direktem Vergleich stimmig mit dem Fazit sein!

\section{Thematische Abgrenzung}

Die vorliegende Arbeit fokussiert sich auf der Prozessebene die direkte Beschaffung von komplexen Bauteilen in der Automobilindustrie und die damit verbundenen Prozesse. Da die Arbeit im Kontext eines Beratungsengagements der SAP SE bei BMW entstanden ist und aufgrund der IT-Strategie des Kunden eine homogene Systemlandschaft angestrebt wird, werden lediglich Umsetzungsmöglichkeiten innerhalb des SAP-Produktportfolios betrachtet. Andere Anbieter und Lösungen werden nicht berücksichtigt, da diese beispielsweise nach Gesichtspunkten der Integration und Masterdatenverfügbarkeit nicht zielführend wären. 

Zudem wird funktional im Allgemeinen die massenhafte Bearbeitung von zentralen Einkaufsverträgen und im Speziellen die Bearbeitung verschiedener Preisbestandteile betrachtet. Weitere Aspekte des Central Contracts werden über Standardfunktionalitäten abgedeckt und sind somit nicht Gegenstand dieser Arbeit.

Aufgrund des limitierten Umfangs der Arbeit wird sich auf die Konzeption der Lösungen beschränkt. Letztere werden lediglich prototypisch umgesetzt, jedoch nicht vollständig implementiert.

% -> Weiterer Abstraktionsgrad auf generelles SAP-Umfeld oder generelles Massendatenmanagement wäre schön für wissenschaftliche Relevanz, aber nur soweit es Thema zulässt, wenn nicht möglich muss das gut begründet werden

\section{Methodisches Vorgehen}

In dieser Arbeit wurde die Methode des Experteninterviews angewende, um die Anforderungen des Kunden an den Prozess der Massenbearbeitung von Zentralen Einkaufsverträgen zu ermitteln. Zu diesem Zweck wurde der Projektmanager für das globale Beschaffungssystem und Verantwortlicher für die Einkaufsprozesse bei BMW interviewt. Experteninterviews ermöglichen in diesem Zusammenhang einen umfassenden Erkenntnisgewinn und Zugang zu praxisrelevanten Informationen, die so in der Literatur nicht verfügbar sind, da es sich um kundenspezifische Anforderungen handelt. Bei der Durchführung wurde eine unstrukturierte Vorgehensweise gewählt, da das allgemeine Ziel die Anforderungsermittlung war und die Anforderungen aus einer unvoreingenommener und nicht durch vordefinierte Fragen eingeschränkter Kundenperspektive aufgenommen werden sollten. Dennoch wurden bei Bedarf in einzelnen Bereichen Rückfragen gestellt, um gezielte Informationen zu erhalten.

Die Bewertung der in der Arbeit vorgestellten Lösungen erfolgt durch eine Nutzwertanalyse. Diese Methode ermöglicht eine quantitative Bewertung der Lösungen anhand von Kriterien, die im Vorfeld definiert wurden. Die Kriterien wurden in Abstimmung mit dem Kunden festgelegt und sollen die Anforderungen des Kunden an die Lösung widerspiegeln. Die Bewertung der Kriterien erfolgt durch den Autor der Arbeit auf einer Skala von eins bis fünf Punkten, woraus eine Gesamtbewertung der Lösungen ermittelt. Somit kann eine fundierte Handlungsempfehlung gegeben werden.
\chapter{Theoretische Grundlagen}

-> Nur Theorie, die später auch verwendet wird, nichts einfach so einführen

-> Voraussetzung: Basiswissen WI-Studium

\section{Geschäftsprozessanalyse und Prozessoptimierung}

-> Allgemeine Theorie zur Geschäftsprozessanalyse und Prozessoptimierung (wenn passende Literatur vorhanden auch direkt in Verbindung mit Massendaten-Management)

-> Darstellung von Methoden/ Frameworks zur Prozessanalyse, -optimierung

\section{User Experience im Geschäftsprozesskontext}

-> Literatur zu UX (allg., Massendaten-Management-Kontext, Business-Software-Kontext)

\section{Massendaten-Management}

-> allgemeine Theorie hinter effizientem Massendaten-Management erläutern (Anlage, Verwaltung, Änderung, Löschung)

\section{SAP Central Procurement insb. Central Contracts}

-> Direct Sourcing Suite von SAP

-> Central Procurement eingehen (Zweck des Produkts, wichtige Features, Anwender, ...)

-> Central Contracts (Was stellt das Objekt im Prozess dar, wie wird es genutzt, welche Daten werden dort abgelegt?, ...)
\chapter{Praktischer Teil}

\section{Globaler Prozesskontext und Ist-Analyse}

\section{Anforderungsanalyse}

\section{Konzeption und Umsetzungsmöglichkeiten des optimierten Prozesses}

\subsection{Anpassung des Standards }

\subsection{Entwicklung einer ma\ss geschneiderten Lösung}

\section{Evaluation der verschiedenen Lösungsansätze}


\chapter{Schlussbetrachtungen}

\section{Zusammenfassung}

-> relevante Punkte der Arbeit zusammenfassen, damit der Leser im nächsten Kapitel der Handlungsempfehlung die wichtigsten Punkte vor Augen hat

\section{Handlungsempfehlung}

-> Hier soll auf Basis der Evaluation der Lösungsansätze anhand der Kriterien eine Handlungsempfehlung gegeben werden, wie der Prozess umgesetzt werden sollte, damit für den Kunden ein Mehrwert entsteht

\section{kritische Reflexion und Ausblick}

-> Zuletzt soll sich hier nochmal kritisch mit der Arbeit auseinandergesetzt werden, eventuell nicht berücksichtigte Punkte, etc. angesprochen werden, damit der Leser ein realistisches Bild erhält

-> Der Ausblick soll das weitere Vorgehen im Kundenprojekt beschreiben (wie die Erkentnisse der Arbeit angewandt werden) und ggf. weiteres Forschungspotenzial auf einem allgemeineren Level aufzeigen

% ---- Literaturverzeichnis
\cleardoublepage
\renewcommand*{\chapterpagestyle}{plain}
\pagestyle{plain}
%\pagenumbering{Roman}                   % Römische Seitenzahlen
%\setcounter{page}{\numexpr\value{savepage}+1}

\printbibliography[title = Literaturverzeichnis]

% ---- Anhang
\appendix
%\clearpage
%\pagenumbering{Roman}  % römische Seitenzahlen für Anhang

\chapter{Anhang}



%\includepdf[pages = -]{Literatur/bgPF_Wiki.pdf}

\newpage
\end{document}
